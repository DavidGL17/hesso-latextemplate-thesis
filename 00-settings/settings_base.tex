% ==================
% Template settings
% ==================

% General tools
% -------------
\usepackage{etoolbox}

% Page style
% ----------
\usepackage[margin=3cm, left=3.5cm, right=3.5cm, twoside=true]{geometry}
\usepackage{fancyhdr}
\setlength{\headheight}{14pt}
\renewcommand{\sectionmark}[1]{\markright{\thesection\ #1}}
\pagestyle{fancy}

% Standard pages (inside chapters)
\fancyhf{}
\renewcommand{\headrulewidth}{0.4pt}
\renewcommand{\footrulewidth}{0pt}
\fancyhead[OR]{\bfseries \nouppercase{\rightmark}}
\fancyhead[EL]{\bfseries \nouppercase{\leftmark}}
\fancyfoot[EL,OR]{\thepage}

% First page of chapters
\fancypagestyle{plain}{
	\fancyhf{}
	\renewcommand{\headrulewidth}{0pt}
	\renewcommand{\footrulewidth}{0pt}
	\fancyfoot[EL,OR]{\thepage}
}

% Imports for external PDFs
\fancypagestyle{addpagenumbersforpdfimports}{
	\fancyhead{}
	\renewcommand{\headrulewidth}{0pt}
	\fancyfoot{}
	\fancyfoot[RO,LE]{\thepage}
}

% Use empty style for page when clearing double pages
\def\cleartoodd{%
	\clearpage%
	\ifodd\value{page}\else\mbox{}\thispagestyle{empty}\newpage\fi%
}

\def\clearchap{%
	\ifodd\value{page}\else\mbox{}\thispagestyle{empty}\fi%
}

% \cleardoublepage replaced by \cleartoodd
\let\origdoublepage\cleardoublepage
\renewcommand{\cleardoublepage}{%
	\cleartoodd%
}

% Fonts
% -----

% Helvetica (Arial used in the MSE Word template)
\usepackage{helvet}

% Math
% ----
\usepackage{amsmath}  % better math

% Floats and figures
% ------------------
\usepackage{newfloat}          % floats
\usepackage[twoside]{caption}  % captions
\usepackage{subcaption}        % subcaptions
\usepackage[section]{placeins} % allows to put float barriers

% Float captions in italics, with label in margin
\DeclareCaptionLabelFormat{title}{#1 #2}
\DeclareCaptionLabelFormat{hangout}{\llap{#1 #2\hspace{5mm}}}
\captionsetup{
	format=hang,
	labelformat=hangout,
	singlelinecheck=false,
	font={it},
    justification=centering
}

% Caption with source for figure
% TODO: improve this to use square brackets like the normal "caption"
\newcommand*{\captionsource}[3]{%
	\caption[{#1}]{%
		#2%
		
		\textbf{Source:} #3%
	}%
}

% Tables
% ------
\usepackage{booktabs} % much better tables
\usepackage{multirow} % allows to fuse rows
\usepackage{array}    % manipulate array
\usepackage{tabularx} % better tables

% Define new tabularx column types:
%  - R: streteched right aligned
%  - C: stretched centered
%  - N: left aligned, specified space
% Not using these, switching to columns where you specify the width
%\newcolumntype{R}{>{\raggedleft\arraybackslash}X}%
%\newcolumntype{C}{>{\centering\arraybackslash}X}%
%\newcolumntype{N}[1]{>{\raggedleft\arraybackslash}p{#1}}

%% Definitions for line breaks in tables
% left fixed width:
\newcolumntype{L}[1]{>{\raggedright\arraybackslash}m{#1}}
% center fixed width:
\newcolumntype{C}[1]{>{\centering\arraybackslash}m{#1}}
% flush right fixed width:
\newcolumntype{R}[1]{>{\raggedleft\arraybackslash}m{#1}}

% Set row height multiplicator to provide more breathing space
\renewcommand{\arraystretch}{1.3} 

% Bibliography
% -------------------

% Use biber, with numeric style and no sorting (citation order)
\usepackage[
backend=biber,
style=numeric,
sorting=none,
bibencoding=auto
]{biblatex}
\addbibresource{03-tail/bibliography.bib}


% Tables of contents, figures, tables and listings
% ------------------------------------------------
\usepackage{tocloft}
\newlistof{listing}{lol}{List of Listings}
\setcounter{tocdepth}{3} % Depth to 'subsubsection'
\setlength{\cftfigindent}{0pt}  % remove indentation from figures in lof
\setlength{\cftfignumwidth}{1cm}
\setlength{\cfttabindent}{0pt}  % remove indentation from tables in lot
\setlength{\cfttabnumwidth}{1cm}
\setlength{\cftlistingindent}{0pt}
\setlength{\cftlistingnumwidth}{1cm}

% Mini tables of contents
% -----------------------
\usepackage{minitoc}

% no "Contents" title
\mtcsettitle{minitoc}{Contents} 

% Layout
\setlength{\mtcindent}{-0.5em}
\mtcsetoffset{minitoc}{-1em}

% Spacing above and below table
\mtcsetfeature{minitoc}{before}{\vspace{0.5cm}}
\mtcsetfeature{minitoc}{after}{\vspace{-0.25cm}}
\renewcommand{\mtifont}{\sffamily\bfseries\large}

% Colors & graphics
% -----------------
\usepackage[table]{xcolor}    % colors
\usepackage[pdftex]{graphicx} % graphics importing
\graphicspath{{02-main/figures/}}
\definecolor{gray80}{gray}{0.80}


% Code and syntax highlighting
% ----------------------------
\usepackage[newfloat]{minted}   % code highlighting

% Typography
% ----------
\usepackage{csquotes}                    % paragraph indentation and spacing
\usepackage[defaultlines=3,all]{nowidow} % avoid widows and orphans
\usepackage{microtype}                   % typographic improvements
\usepackage{parskip}                     % No indent and auto-space between paragraphs
\usepackage[super]{nth}

\usepackage{paralist}
\usepackage{enumitem}
\setlist{after=\vspace{\baselineskip}}

% Section and chapters headings
% -----------------------------
\usepackage[explicit]{titlesec} % titles formatting
%\usepackage{titletoc} % titles formatting in ToC etc
%\usepackage{sectsty}  % sectioning commands

% -- Chapters --
% Remove "Chapter N" and use a sans-serif font

% Set layout lengths
\setlength{\headheight}{8mm}
\setlength{\footskip}{1.5cm}
\addtolength{\textheight}{-.5cm}

% Set numbering
\setcounter{secnumdepth}{3}

\titlespacing{\chapter}{-5mm}{-10mm}{3mm}
\titlespacing{\section}{-5mm}{3mm}{2mm}
\titlespacing{\subsection}{-5mm}{2mm}{2mm}
\titlespacing{\subsubsection}{-5mm}{2mm}{1mm}
\titlespacing{\subsubsubsection}{-5mm}{2mm}{1mm}


%\titleformat{\chapter}[block]
%{\Huge}
%{\thechapter\hspace{12pt}\textcolor{gray80}{|}\hspace{12pt}}
%{0pt}
%{\Huge\bfseries}

\titleformat{\chapter}{\Huge\bfseries}{\llap{\thechapter\hspace{12pt}\textcolor{gray80}{|}}}{0mm}{%
	\hfill\begin{minipage}[t]{\dimexpr\textwidth}\raggedright#1\end{minipage}%
}
\titleformat{\section}{\Large\bfseries}{\llap{\thesection}}{0mm}{%
	\hfill\begin{minipage}[t]{\dimexpr\textwidth}\raggedright#1\end{minipage}%
}
\titleformat{\subsection}{\large \bfseries}{\llap{\thesubsection}}{0mm}{%
	\hfill\begin{minipage}[t]{\dimexpr\textwidth}\raggedright#1\end{minipage}%
}
\titleformat{\subsubsection}{\bfseries}{\llap{\thesubsubsection}}{0mm}{%
	\hfill\begin{minipage}[t]{\dimexpr\textwidth}\raggedright#1\end{minipage}%
}


% Misc
% ------
\usepackage{lipsum}    % filler text
\usepackage{blindtext} % random text
\usepackage{lscape}    % easy landscape pages
\usepackage{pdflscape} % landscape pages for PDFs

% Allow email typesetting
\newcommand{\email}[1]{%
	\href{mailto:#1}{\textit{#1}}%
}

% References
% -----------
\usepackage{url}

% pdf metadata
\usepackage[
	pdfauthor={\Author},
	pdftitle={\ThesisTitle},
	pdfsubject={\ThesisSubject},
	pdfkeywords={\Keywords}
	pdfduplex=DuplexFlipLongEdge]{hyperref}
		
% Hyperlinks
\hypersetup{
	colorlinks=true,
	linkcolor=black,
	citecolor=black,
	filecolor=black,
	urlcolor=black,
}
\providecommand*{\listingautorefname}{Listing}


% Glossary
% --------
\usepackage[xindy,toc]{glossaries}
% Terms
% -----
% format:  \newglossaryentry{<label>}{<settings>}
% example: \newglossaryentry{computer}
%{
%	name=computer,
%	description={is a programmable machine that receives input,
%		stores and manipulates data, and provides
%		output in a useful format}
%}
\newglossaryentry{nosql}
{
	name=NoSQL,
	description={Database not using the relational model and the \acrshort{sql} language}
}

% Acronyms
% --------
% format:  \newacronym{<label>}{<abbrv>}{<full>}
% example: \newacronym{lvm}{LVM}{Logical Volume Manager}
% plural:  \newacronym[longplural={Frames per Second}]{fpsLabel}{FPS}{Frame per Second}

\newacronym{api}{API}{Application Programming Interface}

\newacronym{cep}{CEP}{Complex Event Processing}
\newacronym{ci}{CI}{Continuous Integration}
\newacronym{cqrs}{CQRS}{Command Query Responsibility Segregation}
\newacronym{crud}{CRUD}{Create-Read-Update-Delete}

\newacronym{dag}{DAG}{Directed Acyclic Graph}
\newacronym{dsl}{DSL}{Domain Specific Language}

\newacronym{eca}{ECA}{Event Condition Action}
\newacronym{elk}{ELK}{Elasticseach Logstash and Kibana}
\newacronym{efk}{EFK}{Elasticseach Fluentd and Kibana}
\newacronym{epa}{EPA}{Event Processing Agent}
\newacronym{epn}{EPN}{Event Processing Network}

\newacronym{gelf}{GELF}{Graylog Extended Log Format}
\newacronym{ge}{GE}{Generic Enabler}

\newacronym{ide}{IDE}{Integrated Development Environment}
\newacronym{iot}{IoT}{Internet of Things}

\newacronym{jar}{JAR}{Java ARchive}
\newacronym{jmx}{JMX}{Java Management Extensions}
\newacronym{json}{JSON}{JavaScript Object Notation}
\newacronym{jvm}{JVM}{Java Virtual Machine}

\newacronym{poc}{PoC}{Proof of Concept}

\newacronym{rest}{REST}{Representational state transfer}
\newacronym{rest_markup}{reST}{reStructuredText}
\newacronym{rpc}{RPC}{Remote Procedure Call}

\newacronym{sql}{SQL}{Structured  Query Language}

\newacronym{uuid}{UUID}{Universally Unique Identifier}
\newacronym{uri}{URI}{Universal Resource Identifier}

\makeglossaries

%for images sideways
\usepackage{rotating}

% For listings
% code
\usepackage{listings}
% code colors
\usepackage{listings}
\usepackage{xcolor}

\definecolor{codegreen}{rgb}{0,0.6,0}
\definecolor{codegray}{rgb}{0.5,0.5,0.5}
\definecolor{codepurple}{rgb}{0.58,0,0.82}
\definecolor{backcolour}{rgb}{0.95,0.95,0.92}

\lstdefinestyle{mystyle}{
    backgroundcolor=\color{backcolour},   
    commentstyle=\color{codegreen},
    keywordstyle=\color{magenta},
    numberstyle=\tiny\color{codegray},
    stringstyle=\color{codepurple},
    basicstyle=\ttfamily\footnotesize,
    breakatwhitespace=false,         
    breaklines=true,                 
    captionpos=b,                    
    keepspaces=true,                 
    numbers=left,                    
    numbersep=5pt,                  
    showspaces=false,                
    showstringspaces=false,
    showtabs=false,                  
    tabsize=2
}

\lstset{style=mystyle}

% To use this : 
% \begin{listing}[!ht]
%     \begin{lstlisting}[language=python, breaklines=true,breakatwhitespace=false]
%     ********CODE*******
%     \end{lstlisting}
%     \caption[test]{test}\label{lst:test}
% \end{listing}